\documentclass{article}
\usepackage[utf8]{inputenc}
\usepackage[siunitx]{circuitikz}
\usepackage{mathtools}

\begin{document}
\setlength{\parskip}{10pt plus 1pt minus 1pt}


\section{CapSense example circuit}

This is the example circuit given for the CapacitanceSensor.h library.

\begin{circuitikz} \draw
(0,2) node[left]{Send} to[R=10<\mega\ohm>, o-] (4,2)
      -- (4,0)
      to[vC, l=$C_{sensed}$, *-o] (0,0) node[left]{Receive}
(4,0) to[C=100<\pico\farad>] (8,0)
(8,0) node[ground] {}
;
\end{circuitikz}


\section{Soil moisture measurement circuit}

This is a proposed circuit for measuring soil moisture. The RC time constant is primarily related to the combination of the 5 M$\Omega$ resistors and $C_{soil}$.

\begin{equation}
\tau = ( 5 M\Omega + 5 M\Omega ) ( C_{soil} )
\end{equation}

A 100 pF capacitor is connected to the receive side of the soil to stabilize the output.

\begin{circuitikz} \draw
(0,3) node[left]{Send} to[R=5<\mega\ohm>, o-] (4,3)
      to[vC, l=$C_{soil}$] (4,0)
      to[R=5<\mega\ohm>, *-o] (0,0) node[left]{Receive}
(4,0) to[C=100<\pico\farad>] (8,0)
(8,0) node[ground] {}
;
\end{circuitikz}

Some experimentation should be done to determine ideal values for the resistors and whether the grounding capacitor is beneficial.


\section{Watering circuit}

This circuit opens a solenoid valve to water plants. Note that the transistors:
\begin{itemize}
\item Q1 is a phototransistor (which comprises half of an optoisolator)
\item Q2 is a darlington transistor
\end{itemize}

Should the grounds be connected?

\begin{circuitikz} \draw
(-4, 4) node[left]{On} to[R, l=$R_1$, o-] (-1,4)
-- (-1, 3) to[leDo] (-1,-1) node[ground] {}

(1,1) node[npn] (npn) {Q1}
(npn.base)
(npn.emitter) -- (1, 0)
(npn.collector) to[R, l=$R_2$] (1, 5) -- (3, 5)

(3,0) node[npn] (npn) {Q2}
(npn.base) -- (1, 0)
(npn.emitter) node[ground] {}
(npn.collector) to[L, -o] (3,6) node[above]{+12 V}

(3, 1) -- (5, 1) to[Do] (5, 5) -- (3, 5)
;
\end{circuitikz}

\begin{equation}
I_B = \frac{I_L}{\beta} = \frac{500 mA}{1000} = 500 \mu A
\end{equation}

Is it correct that $V_{CE,Q1,sat}$ is 0 V?

\begin{equation}
R_{2} = \frac{ V_{R2} }{ I_{B2} } = \frac{ 12 V - V_{CE,Q1} - V_{BE,Q2} }{ I_{B2} } = \frac{ 12 V - 0 V - 1.4 V }{ 500 \mu A } = \frac{ 10.6 V }{ 500 \mu A } = 21.2 k\Omega
\end{equation}

\begin{equation}
R_{1} = \frac{ V_{on} - V_{diode} }{ I_{diode} } = \frac{ 5 V - 2.7 V }{ I_{diode} }
\end{equation}


\section{Tank level sensor}

This circuit detects the level of water in the reservoir.

\begin{circuitikz}
\draw
(3,0) -- (3,6) to[closing switch, *-] (6,6) to[R, l=$R_{in}$, -o] (8,6) node[right]{L2} to [R, l=$R_{pull-up}$, -o] (8,8) node[above] {+5 V}
(3,0) -- (3,3) to[closing switch, *-] (6,3) to[R, l=$R_{in}$, -o] (10,3) node[right]{L1} to [R, l=$R_{pull-up}$, -o] (10,5) node[above] {+5 V}
(0,0) node[left]{On} to[R, l=$R_{out}$, o-] (3,0) to[closing switch, *-] (6,0) to[R, l=$R_{in}$, -o] (12,0) node[right]{L0} to [R, l=$R_{pull-up}$, -o] (12,2) node[above] {+5 V}
;
\end{circuitikz}

L0, L1, and L2 can be fed through a priority encoder to provide 4 water levels on 2 lines.

\begin{tabular}{c{c}|c}
  D0 & D1 & Water Level \\
  \hline
  0 & 0 & Empty \\
  0 & 1 & Low \\
  1 & 0 & Mid \\
  1 & 1 & Full \\
\end{tabular}

\end{document}
